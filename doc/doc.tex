% Created 2019-11-24 Sun 14:30
% Intended LaTeX compiler: pdflatex
\documentclass[11pt]{article}
\usepackage[utf8]{inputenc}
\usepackage[T1]{fontenc}
\usepackage{graphicx}
\usepackage{grffile}
\usepackage{longtable}
\usepackage{wrapfig}
\usepackage{rotating}
\usepackage[normalem]{ulem}
\usepackage{amsmath}
\usepackage{textcomp}
\usepackage{amssymb}
\usepackage{capt-of}
\usepackage{hyperref}
\usepackage{minted}
\usepackage[margin=2cm]{geometry}
\usepackage[brazil, ]{babel}
\setlength{\parindent}{0cm}
\hypersetup{colorlinks=true,linkcolor=blue,urlcolor=blue}
\usepackage{enumitem}
\usepackage[pt-BR]{datetime2}
\DTMlangsetup{showdayofmonth=false}
\author{Gabriel Bastos -- 2016058204}
\date{\today}
\title{Redes -- TP3}
\hypersetup{
 pdfauthor={Gabriel Bastos -- 2016058204},
 pdftitle={Redes -- TP3},
 pdfkeywords={},
 pdfsubject={},
 pdfcreator={Emacs 26.3 (Org mode 9.2.6)}, 
 pdflang={Pt_Br}}
\begin{document}

\maketitle
\tableofcontents


\section{Proposta}
\label{sec:orgb36fa34}
Com o objetivo de adquirir experiência e conhecimento em aplicações reais que lidam com
redes, extendi os requisitos do trabalho de forma a se assemelhar mais com um programa
real, que lida com diversas condições de rede e usuários sob os quais não se têm
controle.
\subsection{Número de clientes}
\label{sec:orge192b72}
O número máximo de clientes será sempre limitado pelo número de portas no protocolo de
rede. Apesar disso, julguei interessante eliminar da aplicação este tipo de restrição.
Recebendo os clientes de forma dinâmica, é possível conter essa limitação aos recursos
do sistema, e não no código. Desta forma, o número de clientes simultâneos não deve ser
limitado por uma constante no programa.
\subsection{Tamanho das mensagens}
\label{sec:org6ea8d31}
O tamanho das mensagens também deve ser dinâmico, de forma a fazer o melhor uso de
banda e memória. Contudo, uma limitação se faz necessária: para evitar que clientes
maliciosos possam executar um ataque \emph{DDoS} através de mensagens muito grandes, um
tamanho máximo deve ser estipulado. Este tamanho deve ser facilmente alterável em tempo
de compilação, de forma a permitir mensagens maiores, caso necessário.
\subsection{Mensagens}
\label{sec:org8e60697}
Aplicações reais devem ser capazes de lidar, em uma única operação de leitura, com
múltiplas mensagens consecutivas, bem como mensagens divididas e incompletas. É
possível que, caso um cliente envie dois pacotes, um sofra atraso e chegue separado. De
forma semelhante, uma congestão na rede pode fazer com que vários pacotes, mesmo que
enviados separadamente, cheguem juntos. O servidor deve ser capaz de lidar com tais
situações, de forma que não ocorra perda de dados.
\subsection{Nome de usuário}
\label{sec:orgaa5fbe6}
Com o objetivo de permitir maior flexibilidade aos clientes, deve ser permitido
usuários anônimos. Estes usuários não serão capazes de receber mensagens diretas, mas
participarão de \emph{broadcasts}, e poderão receber a lista de usuários conectados.
\section{Projeto}
\label{sec:orgf629313}
De forma a se obter um melhor aproveitamento dos recursos computacionais, o servidor
utiliza a chamada de sistema \texttt{poll} para lidar com múltiplos clientes. Optei por essa
chamada por ser uma versão moderna e robusta da chamada \texttt{select}, indicada na
especificação original.
\subsection{Mensagens}
\label{sec:orgf3bd14f}
Para permitir mensagens consecutivas, se faz necessária a delimitação destas. Optei
por utilizar os caracteres de controle da table unicode \footnote{\href{https://en.wikipedia.org/wiki/List\_of\_Unicode\_characters\#Control\_codes}{Unicode control codes}}
como delimitadores. Os caracteres utilizados são:
\begin{center}
\begin{tabular}{ll}
Caractere & Significado\\
\hline
\texttt{SOH} & Start of Heading\\
\texttt{STX} & Start of Text\\
\texttt{IND} & Index\\
\texttt{EOT} & End of transmission\\
\texttt{ENQ} & Enquiry character\\
\texttt{NAK} & Negative acknowledge\\
\texttt{US} & Unit Separator\\
\texttt{PM} & Private Message\\
\end{tabular}
\end{center}
\subsubsection{Cliente $\rightarrow$ Servidor}
\label{sec:org063775e}
\begin{description}
\item[{Definição de nome:}] Mensagem para definir o nome do usuário. \\
Um nome vazio indica anonimidade.
\begin{verbatim}
SOH IND <name> EOT
\end{verbatim}
\item[{Lista de usuários:}] Mensagem para requerer a lista de usuários.
\begin{verbatim}
SOH ENQ EOT
\end{verbatim}
\item[{\emph{Broadcast}:}] Mensagem de texto para todos usuários conectados.
\begin{verbatim}
SOH STX <texto> EOT
\end{verbatim}
\item[{\emph{Unicast}:}] Mensagem de texto para um usuário específico.
\begin{verbatim}
SOH PM <destinatário> STX <texto> EOT
\end{verbatim}
\end{description}
\subsubsection{Servidor $\rightarrow$ Cliente}
\label{sec:org56e51bb}
\begin{description}
\item[{Erro:}] Mensagem de erro, resposta à uma requisição inválida do cliente.
\begin{verbatim}
SOH NAK <código de erro> EOT
\end{verbatim}

\begin{tabular}{rl}
Código de erro & Significado\\
\hline
0x01 & Nome inválido (em uso)\\
0x02 & Destinatário inválido\\
\end{tabular}
\item[{Lista de usuários:}] Mensagem com a lista de usuários.
\begin{verbatim}
SOH ENQ <usuário> US <usuário> US <usuário> ... EOT
\end{verbatim}
\item[{Texto:}] Mensagem de texto.
\begin{verbatim}
SOH PM <remetente> STX <texto> EOT
\end{verbatim}
\end{description}
\subsection{Destinatário}
\label{sec:orgd78be13}
Para a identificação rápida do destinatário em mensagens \emph{unicast}, um índice de clientes
foi implementado. Desta forma, não é necessário buscar na coleção de clientes o alvo
da mensagem.
\section{Comandos}
\label{sec:org91bcaeb}
\subsection{Desconectar}
\label{sec:orgd1b5042}
Comando:
\begin{verbatim}
exit
\end{verbatim}
\subsection{Nome de usuário}
\label{sec:orgbccc360}
Ao entrar no servidor, o usuário é inicialmente anônimo. Seu nome pode ser alterado a
qualquer momento, quantas vezes desejar.

Comando:
\begin{verbatim}
name;<nome>
\end{verbatim}

Para retornar à anonimidade:
\begin{verbatim}
name;
\end{verbatim}
\subsection{Listar usuários}
\label{sec:org7f06a73}
Comando:
\begin{verbatim}
users
\end{verbatim}
\subsection{Mensagem \emph{broadcast}:}
\label{sec:orgac04490}
Comando:
\begin{verbatim}
all;<mensagem>
\end{verbatim}
\subsection{Mensagem \emph{unicast}:}
\label{sec:org96fdd29}
Comando:
\begin{verbatim}
uni;<destinatário>;<mensagem>
\end{verbatim}
\end{document}
